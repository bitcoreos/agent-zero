Antibody Self-Association: AC-SINS Assay, Modeling, and Formulation Strategies
Affinity-Capture Self-Interaction Nanoparticle Spectroscopy (AC-SINS)
Assay Principle & Readout
AC-SINS is a high-throughput assay that probes antibody self-interactions at low concentrations by using gold nanoparticles (AuNPs) coated with capture antibodies (e.g. anti-human Fc)
pmc.ncbi.nlm.nih.gov
. Monoclonal antibodies (mAbs) from dilute solution bind to these AuNP conjugates; if the mAbs have attractive self-interactions, they effectively cross-link the AuNPs into clusters, bringing particles closer together
researchgate.net
. This clustering leads to a red-shift of the AuNP surface plasmon resonance absorbance peak (typically from a baseline ~530 nm for well-dispersed 20 nm AuNPs to ~574–600 nm for aggregated conjugates)
researchgate.net
. The magnitude of this red-shift (denoted Δλ_max) serves as a quantitative readout of self-association propensity. Little or no shift indicates minimal self-interaction, whereas a pronounced shift toward longer wavelengths indicates significant self-association. Δλ_max values are measured by UV–Vis spectroscopy (scanning, for example, 450–750 nm) and are calculated relative to a non-self-associating reference or a no-mAb conjugate control
patents.google.com
patents.google.com
. Typical assay mAb concentrations are in the low μg/mL range (e.g. 5–50 μg/mL), enabling early developability screening when material is limited
omniab.com
.
Behavior in PBS (pH 7.4)
Historically, AC-SINS has been performed in phosphate-buffered saline at pH 7.4 (near physiological conditions) as a default condition, which allows high-throughput screening with consistent baseline conditions
researchgate.net
. Under pH 7.4 PBS, the rank-order Δλ_max values for a panel of antibodies tend to correlate with their tendency to self-associate and even with in vivo behavior like nonspecific clearance
researchgate.net
researchgate.net
. In discovery campaigns, candidates showing low shifts in PBS are often enriched for desirable low-viscosity and low-aggregating properties in downstream high-concentration formulations
omniab.com
. However, the predictive power of PBS-screened Δλ_max for high-concentration formulation behavior is formulation-dependent. In cases where the final drug product uses a different pH or excipients (e.g. a histidine buffer at pH 6), the PBS assay’s correlation with actual viscosity or stability can weaken
researchgate.net
. Aligning the assay buffer with the target formulation conditions improves the relevance of the readout
researchgate.net
researchgate.net
. In practice, PBS AC-SINS is an efficient primary screen to flag molecules with extreme self-association, but follow-up in a formulation-matched assay (or orthogonal methods) is recommended for candidates near risk thresholds
researchgate.net
researchgate.net
.
Operational Specifics
In a typical AC-SINS setup, ~20 nm diameter AuNPs are used as the solid support
link.springer.com
. A polyclonal anti-Fc capture antibody (commonly goat anti-human IgG Fc) is adsorbed onto the AuNP surface, and the remaining surface is blocked with a thiol-PEG reagent to stabilize the conjugates and prevent non-specific aggregation
academic.oup.com
. This produces an immunogold reagent with a baseline plasmon peak around 520–530 nm in the absence of test mAbs
researchgate.net
. To run the assay, the conjugate is mixed with the test mAb sample (in the chosen buffer) and incubated briefly (minutes to hours as needed) at ambient conditions. UV–Vis absorbance spectra are then recorded (typically scanning from ~450 to 750 nm) for each sample and for appropriate controls
patents.google.com
patents.google.com
. Δλ_max is determined by comparing the peak wavelength of the mAb sample conjugate to that of a reference (such as the conjugate in buffer without mAb, or a well-behaved mAb conjugate under the same conditions)
patents.google.com
patents.google.com
. To ensure data quality, it is important to validate each batch of conjugates (confirming the baseline peak ~530 nm and stability over the assay duration) and to run internal control mAbs known to have low or high self-association as benchmarks. Common positive/negative controls include trastuzumab (low self-interaction, minimal shift) and bococizumab or sirukumab (high self-interaction, large shift)
pmc.ncbi.nlm.nih.gov
. Including such controls allows normalization or at least qualitative comparison across plates and runs.
Empirical Risk Bands in PBS (Screening Guidance)
Experience from large panels of mAbs has led to empirical “risk band” criteria for Δλ_max (in PBS pH 7.4) to guide developability decisions
omniab.com
omniab.com
:
Δλ_max < 5 nm: This is considered a favorable result. Shifts below ~5 nm are within noise or baseline variation for the assay and indicate a very low degree of self-association. Many well-behaved clinical-stage or approved antibodies exhibit Δλ_max in this range
omniab.com
. A stringent cutoff of ~5 nm has been used in discovery to select only molecules with minimal self-interaction for progression
omniab.com
omniab.com
.
Δλ_max ~5–10 nm: Shifts in this intermediate range are a cautionary zone. They suggest some propensity for self-interaction, though not as extreme as known problematic mAbs. Molecules in this band might still be developable, but it is advisable to investigate further. Orthogonal assays – such as diffusion interaction parameter (k_D) measurement, cross-interaction chromatography, or polyethylene glycol precipitation (see below) – can be employed to confirm whether the antibody’s behavior is truly borderline or if the AC-SINS result might be buffer-specific. Sequence analysis to identify surface patches (e.g. hydrophobic or charged regions) that could be driving self-association is also often undertaken for molecules in this zone.
Δλ_max > 10 nm: High shifts above 10 nm indicate a significant self-association liability. Values in this range are frequently observed with antibodies known to have developability issues (for example, some antibodies that showed very high viscosity or aggregation in clinical formulation studies were later found to have large AC-SINS shifts in screening)
omniab.com
. Such antibodies often require engineering (e.g. point mutations in complementarity-determining regions to remove aggregation-prone patches) to remain viable, or they may be deprioritized in favor of more intrinsically stable candidates. In practice, a Δλ_max of >10 nm (especially in the high teens or beyond) in PBS is a strong red flag for developability risk – indeed, early studies noted that AC-SINS shifts exceeding ~11–12 nm were rarely seen in antibodies that advanced successfully to the clinic
tandfonline.com
.
It should be noted that these numerical cutoffs are not absolute across all projects – they serve as general guidelines. Some teams may apply slightly different thresholds (for instance, using 3–4 nm as a very conservative pass criterion, or tolerating up to ~8 nm if other properties are excellent). Nonetheless, the <5 nm and >10 nm benchmarks are well-supported by published analyses correlating Δλ_max with downstream behaviors
omniab.com
omniab.com
.
Assay Variants and Buffer Dependence
Because AC-SINS results can be buffer-dependent, several variant methods have been developed to extend its utility beyond the standard PBS conditions. One challenge is that certain formulation buffers (notably histidine at mildly acidic pH) can destabilize the AuNP–antibody conjugates or alter the assay baseline, confounding the interpretation of Δλ_max
researchgate.net
. For example, histidine pH 6.0 can itself cause slight AuNP aggregation in conventional AC-SINS, making even a non-self-interacting antibody appear to give a shifted reading. To address this, PEG-stabilized SINS (PS-SINS) was introduced
researchgate.net
. In PS-SINS, the AuNPs are additionally functionalized with a dense PEG layer (e.g. by thiolated PEG) to improve conjugate stability in a wider range of buffers. Phan et al. demonstrated that with PEG stabilization, the assay could reliably be run in histidine buffer (pH 5–6) and acetate buffer, thereby allowing buffer-matched self-association screening in conditions closer to many therapeutic formulations
researchgate.net
. This innovation enables high-throughput screening of mAbs in their formulation-relevant pH without artifactual aggregation of the nanoparticles
researchgate.net
. Another extension is salt-gradient AC-SINS (SGAC-SINS), which profiles the protein’s self-interaction behavior across a range of ionic strengths. Instead of a single buffer condition, the assay is repeated at multiple salt concentrations (e.g. low salt, physiological ~150 mM, high salt conditions). The resulting Δλ_max vs. salt profile yields a “dynamic range” of self-association propensity
patents.google.com
patents.google.com
. Attractive interactions that are primarily electrostatic in nature will strengthen at higher salt (due to screening of charge repulsion)
patents.google.com
, whereas primarily hydrophobic-driven association might be less salt-dependent or even decrease if high salt stabilizes the monomers. By examining the salt-response curve, formulators can glean insight into the nature of the interactions and identify conditions where the antibody is more “repulsive” (e.g. maybe at a specific moderate salt level, the Δλ_max falls below a threshold
patents.google.com
patents.google.com
). SGAC-SINS thus informs the selection of formulation excipients and ionic strength that minimize self-association
patents.google.com
patents.google.com
. A related variant, charge-stabilized SINS (CS-SINS), uses a poly-L-lysine cationic coating on nanoparticles to create a baseline repulsive surface charge, allowing measurement of even strongly attractive antibodies by preventing immediate irreversible aggregation
researchgate.net
researchgate.net
. CS-SINS has been shown to correlate well with viscosity in pH 6.0 histidine formulations, particularly for IgG1 and IgG2 subclasses
researchgate.net
researchgate.net
. In summary, these assay variants (PS-SINS, CS-SINS, salt-gradient SINS) broaden the applicability of nanoparticle-based self-interaction measurements, enabling more relevant screening when an antibody’s formulation will differ from the standard PBS, or when more granular insight into interaction forces is needed.
Complementary Colloidal Metrics
No single assay completely predicts high-concentration behavior, so AC-SINS is often combined with other colloidal stability metrics to build a more robust developability picture
researchgate.net
researchgate.net
. Key complementary techniques include:
Diffusion Interaction Parameter (k_D): This parameter is obtained from dynamic light scattering (DLS) by measuring the concentration-dependence of the antibody’s diffusion coefficient in solution. Practically, DLS plate readers can measure the apparent hydrodynamic size of the mAb across a range of concentrations (e.g. 1–10 mg/mL); the slope of the inverse diffusion coefficient vs. concentration gives k_D
unchainedlabs.com
technologynetworks.com
. A positive k_D indicates that the molecules diffuse faster with increasing concentration, implying net repulsive inter-molecular forces (good colloidal stability), whereas a negative k_D means diffusion slows down as concentration increases – a sign of attractive interactions and incipient self-association
azom.com
. In essence, positive k_D = protein wants to stay monomeric (or even experiences slight net excluded-volume repulsion), and negative k_D = protein tends to cluster transiently (harbinger of aggregation or high viscosity)
azom.com
. k_D is closely related to the osmotic second virial coefficient (B<sub>22</sub>) – both are measures of protein-protein interactions in solution
azom.com
. Studies have shown that k_D measured in the actual formulation buffer is a strong predictor of high-concentration behavior: for example, antibodies with strongly negative k_D in formulation are very likely to exhibit high viscosity or aggregation at 100+ mg/mL
pubmed.ncbi.nlm.nih.gov
pubmed.ncbi.nlm.nih.gov
. High-throughput DLS instruments allow k_D screening with low sample usage (well under 1 mg per condition), making this an attractive orthogonal metric alongside AC-SINS. In fact, one recent work found that low-concentration k_D and Δλ_max together could correctly classify a large majority of antibodies as “high” or “low” viscosity risk, using a threshold like 20 cP at 150 mg/mL as the criterion
mdpi.com
.
Cross-Interaction Chromatography (CIC): In CIC (also known as protein AIC, antibody self-interaction chromatography, or clone self-interaction by biolayer interferometry in some implementations), the antibody’s tendency to interact with a surrogate surface or with immobilized proteins is measured
researchgate.net
researchgate.net
. A common format is to immobilize a polyclonal human IgG (or the Fc region) on a column and then assess the retention time (or binding profile) of the mAb of interest on that column (the mAb can interact with the immobilized IgG via its variable regions)
pubmed.ncbi.nlm.nih.gov
. Longer retention or stronger binding indicates more self-interaction propensity. This method, effectively a chromatographic measurement of nonspecific binding, has been shown to correlate with self-association tendencies. It is a quick, small-sample technique (when done in HPLC format) and has been used to triage large panels
pmc.ncbi.nlm.nih.gov
. For example, Hedberg et al. reported using CIC to rank-order mAbs and found general agreement with AC-SINS results
pmc.ncbi.nlm.nih.gov
. Because CIC can be run at various pH or with different immobilized ligands, it’s a flexible tool. Newer high-throughput versions include batch adsorption assays or bio-layer interferometry methods that similarly measure antibody self- or cross-interactions in a parallel format.
PEG-Induced Precipitation (PEG C<sub>1/2</sub>): Polyethylene glycol (PEG) is a crowding agent that can precipitate proteins out of solution; the concentration of PEG required to precipitate 50% of the protein (the PEG C<sub>1/2</sub>) is inversely related to the protein’s solubility. A high PEG C<sub>1/2</sub> (meaning it takes a lot of PEG to crash out the protein) suggests good solubility, whereas a low PEG C<sub>1/2</sub> indicates the protein is on the edge of solubility/aggregation
pubmed.ncbi.nlm.nih.gov
researchgate.net
. PEG precipitation assays have been used as another orthogonal screen for antibody developability
nature.com
. In practice, the mAb is mixed with increasing concentrations of PEG (e.g. 0–20%) and the absorbance or turbidity is measured to detect onset of precipitation; from the curve, one extracts the C<sub>1/2</sub>. While PEG C<sub>1/2</sub> often correlates with self-association trends (antibodies that self-associate strongly in AC-SINS or have negative k_D often precipitate at lower PEG%), it is not a universal predictor
www-vendruscolo.ch.cam.ac.uk
researchgate.net
. Some antibodies with peculiar interaction profiles might behave differently under PEG vs. physiological conditions
nature.com
researchgate.net
. Therefore, PEG-induced solubility data are interpreted in context with other assays. Nonetheless, PEG precipitation is appealing as it is inexpensive and simple, and recent work even reported open-source automated platforms for PEG screening to integrate into early workflows
sciencedirect.com
sciencedirect.com
.
By combining AC-SINS with these additional metrics, scientists can paint a more complete picture of an antibody’s colloidal stability. For instance, an antibody that shows a mild AC-SINS shift but strongly negative k_D and low PEG C<sub>1/2</sub> would be flagged despite the moderate Δλ_max. Conversely, an antibody with a slightly elevated Δλ_max might be given a pass if k_D, CIC, and PEG assays all indicate low risk. Multivariate approaches (sometimes even employing machine learning on a panel of assay readouts) are increasingly used to make go/no-go decisions in lead optimization
researchgate.net
researchgate.net
.
Link to Viscosity and High-Concentration Behavior
A central goal of early self-association screening is to predict which antibodies will formulate well at the high concentrations required for therapeutic dosing (often 100–200 mg/mL for subcutaneous injection). Both AC-SINS and k_D are dilute-concentration measurements, yet numerous studies have found that they correlate with high-concentration properties like viscosity and aggregation propensity – provided the measurements are done in comparable buffer conditions
pubmed.ncbi.nlm.nih.gov
pubmed.ncbi.nlm.nih.gov
. For example, an antibody that gives a large Δλ_max in a formulation-matched SINS assay and has a strongly negative k_D in that same buffer almost always exhibits high solution viscosity at concentration (and often visible opalescence or phase separation)
pubmed.ncbi.nlm.nih.gov
pubmed.ncbi.nlm.nih.gov
. Conversely, antibodies with little to no plasmon shift and positive k_D tend to remain reasonably low-viscosity (e.g. <10–20 cP) even at 150 mg/mL
pubmed.ncbi.nlm.nih.gov
mdpi.com
. Avery et al. (2018) and others have shown that selecting for low Δλ_max and high k_D during lead selection enriched the pool for molecules that could be formulated at high concentration with acceptable viscosity
omniab.com
omniab.com
. However, an important nuance is that the buffer matters: if your final formulation is pH 5.5 histidine, an AC-SINS result in pH 7.4 PBS may not fully predict the viscosity in histidine (some antibodies flip behavior between buffers due to charge state changes)
researchgate.net
. This has motivated the use of PS-SINS (for histidine) and other condition-matched assays as mentioned. When such matched-condition data are obtained, the correlations can be quite strong. For instance, a recent high-throughput study (Zarzar et al. 2023) reported that Δλ_max in pH 6.0 histidine correlated with 150 mg/mL viscosity (Spearman ρ ~0.8) for a set of IgG1 molecules
researchgate.net
researchgate.net
. Similarly, Bhandari et al. (2023) found that by using both AC-SINS and DLS (k_D) data, they could predict whether an antibody would exceed 20 cP at high concentration with high accuracy
mdpi.com
mdpi.com
. 20 cP is a practical upper limit for subcutaneous injectability with standard needles (above ~20 cP, injection either requires too much force or too large a needle gauge) and is thus often used as a threshold in developability studies
mdpi.com
. Antibodies meeting the “low-risk” criteria (Δλ_max small, k_D positive) generally stay below this viscosity cutoff in formulation, whereas those that fail one or both criteria frequently exceed it
mdpi.com
mdpi.com
. It is worth noting that there are occasional outliers – for example, molecules with moderate self-interaction signals that still behave well at 150 mg/mL, or vice versa. These can occur due to specific interaction geometries or reversible oligomerization that is not easily detected by the low-concentration methods. Therefore, while AC-SINS and related assays greatly de-risk a portfolio, they do not eliminate the need for actual high-concentration stability studies on late-stage leads. They are screening tools to stack the odds in your favor. In summary, though, a strong qualitative alignment exists: antibodies with sticky patches (often manifesting as high Δλ_max, negative k_D, low PEG C<sub>1/2</sub>, etc.) are far more likely to exhibit high viscosity, opalescence, or aggregation when concentrated
pnas.org
. This connection underlies why these early assays have become routine in antibody discovery.
Failure Modes & Controls
AC-SINS is a reliable assay, but several technical factors can lead to misleading results if not properly controlled. One issue is the quality and consistency of the AuNP–anti-Fc conjugate. Variations in the lot of gold nanoparticles (size distribution, surface chemistry) or the capture antibody coating (different species or vendors of anti-Fc, or different coating densities) can shift the baseline plasmon wavelength or alter the sensitivity of the assay
patents.google.com
patents.google.com
. For example, an over-crowded conjugate might sterically hinder antibody binding or self-association, giving falsely low Δλ_max readings. It is therefore important to standardize the conjugation protocol and perform QC on each conjugate batch – checking the baseline λ_max (~530 nm), the conjugate stability (no drift in λ_max over the assay timeframe in blank buffer), and the response to positive/negative control mAbs. Including controls like a known “sticky” mAb and a known “clean” mAb on each plate is recommended to verify that the assay is performing as expected (the sticky control should give a large red shift, the clean control minimal shift)
omniab.com
omniab.com
. Another confounding factor is the buffer composition. As noted, certain buffers (notably those containing imidazole, like histidine) or excipients can destabilize the colloid. High concentrations of salts or polysorbates might also affect the assay if they cause changes in nanoparticle surface properties. It’s good practice to run blank conjugate in each buffer of interest to see if the baseline shifts. If a buffer alone causes a shift, then results in that condition must be interpreted with caution or a stabilized protocol (e.g. PS-SINS) should be used
researchgate.net
. Non-specific aggregation or precipitation can also occur in some cases. If an antibody tends to aggregate on its own even at the low μg/mL levels (which is rare but possible for extremely aggregation-prone clones), it might precipitate or form large complexes that don’t produce the classical shift (they might even scatter light and give an irregular spectrum). Thus, if an AC-SINS result is anomalous (e.g. odd spectral shape), one should check the sample for precipitation or large aggregates (analytical ultracentrifugation or simply a visual inspection for turbidity). In general, AC-SINS is most sensitive to reversible colloidal associations; irreversible aggregate formation is a separate phenomenon and should be screened by other methods (e.g. accelerated stability tests). Finally, replicate measurements and statistical analysis are important. Because Δλ_max differences of a few nanometers can separate “good” from “bad” in a stringent screen, one should ensure the assay is reproducible to ~1 nm or better. Running duplicates or triplicates and averaging the results (or alternatively, using multiple independent conjugate preparations) increases confidence. If an antibody is borderline (say ~6 nm shift), a single experiment might not be definitive – further characterization is warranted rather than a hard acceptance or rejection. AC-SINS data should be considered in context of its error bars and in conjunction with other assays. By understanding these failure modes and implementing appropriate controls and repeat measurements, one can largely mitigate the risks of false positives or negatives in AC-SINS. When properly executed, the assay’s precision is sufficient to discriminate even small differences in self-association propensity across large panels of antibodies
pmc.ncbi.nlm.nih.gov
omniab.com
.
Mechanistic Basis of Self-Interaction
What makes one antibody give a high Δλ_max (or high viscosity) and another behave well? Mechanistic studies and sequence analyses have pointed to certain surface features on the antibody variable domains as primary drivers. In particular, large solvent-exposed hydrophobic patches on the antibody (often in or near the complementarity-determining regions, CDRs) can promote reversible self-association by hydrophobic attraction
pnas.org
pnas.org
. Likewise, positively charged patches (excess cationic surface area) can cause attraction via electrostatic complementarity to negatively charged regions on another antibody, or by binding to acidic excipients, etc. Jain et al. (2017) found that clinical-stage antibodies with unusually high self-association tended to have one or more CDR loops that were both hydrophobic and highly basic, creating what they termed “sticky patches”
pnas.org
. Such patches can mediate non-specific antibody-antibody contacts, leading to clustering in AC-SINS or high viscosities at concentration. Conversely, antibodies with more even surface charge distribution (or a net negative charge at formulation pH) and fewer hydrophobic patches tend to be very colloidally stable
pnas.org
. It has been observed that many developable mAbs carry a slight excess of negative charge (often from acidic residues in frameworks or CDR) and avoid large unbroken hydrophobic regions on their surfaces. Hydrophobic interaction chromatography (HIC) and other assays corroborate that antibodies with high HIC retention (i.e. very hydrophobic surfaces) often show large AC-SINS shifts and poor solubility
pnas.org
pnas.org
. In essence, nonspecific self-interaction is an intermolecular extension of an antibody’s tendency for nonspecific binding – the same traits that make an antibody stick to unrelated antigens or surfaces will make it stick to itself. Indeed, engineering efforts to reduce self-association often simultaneously reduce polyspecific binding. Common strategies include mutating key solvent-exposed hydrophobic residues to polar ones, or introducing charged residues to break up patches. These changes can markedly improve AC-SINS and k_D results, thereby reducing high-concentration viscosity
pubmed.ncbi.nlm.nih.gov
pubmed.ncbi.nlm.nih.gov
. It’s important to note that not all self-association is driven by the Fv region; sometimes interactions involve the constant domains (e.g. idiotype-anti-idiotype interactions in IgGs, or domain swapping). But in most cases identified so far, the primary liabilities trace back to the variable domains. Tools like structure visualization, in silico hot-spot prediction, and even experimental methods like hydrogen-deuterium exchange mass spectrometry (HDX-MS) have been used to pinpoint interaction-prone patches
pubmed.ncbi.nlm.nih.gov
pubmed.ncbi.nlm.nih.gov
. For example, a recent study used HDX-MS to identify regions of a particular IgG1 that became protected upon self-association (implicating them in the interface), and then designed mutations in those regions to disrupt the interaction
pubmed.ncbi.nlm.nih.gov
pubmed.ncbi.nlm.nih.gov
. The mutants showed reduced Δλ_max and much lower viscosity, confirming the mechanistic hypothesis
pubmed.ncbi.nlm.nih.gov
pubmed.ncbi.nlm.nih.gov
. Those kinds of studies reinforce that CDR loops rich in hydrophobic or aromatic residues, or containing clusters of positive charge, are usual suspects for self-interaction sites
pnas.org
. In summary, the mechanistic picture is that self-associating antibodies often present “bad surface chemistry” – patches of nonpolar or mismatched charge that favor like-with-like association. These features manifest in assays as described: elevated AC-SINS shifts, more negative k_D (since attractive forces dominate), and higher likelihood of aggregation or high viscosity. Recognizing these signatures early allows for sequence optimization to improve developability.
Developability Decision Framework
Using the data from AC-SINS and complementary assays, one can establish a decision-making framework to triage antibody candidates. A simple rule-set, assuming we have both Δλ_max (in formulation-matched conditions if possible) and k_D measured, could be:
If Δλ_max is very low (e.g. < 5 nm) and k_D is positive: Advance the candidate. It has clear signals of good behavior (minimal self-association, net repulsive interactions). Such a molecule is likely to formulate well, and can proceed to further development or high-concentration studies
omniab.com
azom.com
.
If Δλ_max is moderate (≈5–10 nm) or k_D is near neutral (around 0): Investigate and engineer. This candidate is borderline. It may be acceptable as-is if other properties (potency, specificity, etc.) are exceptional, but it would be wise to attempt improvements. Protein engineering can be applied – for instance, identifying problematic CDR residues and mutating them to reduce self-interaction. After engineering (or sometimes formulation adjustment), the variant should be re-tested. Many companies at this stage will generate a few point mutants in the Fv region and rescreen; often a single mutation can drop Δλ_max by several nm or turn k_D from slightly negative to positive
pubmed.ncbi.nlm.nih.gov
pubmed.ncbi.nlm.nih.gov
. Only if the issue is corrected (and no major new liabilities are introduced) would the molecule advance. If not, it might be relegated behind other leads.
If Δλ_max is high (> 10 nm) or k_D is clearly negative: Deprioritize or re-engineer extensively. Such a molecule has a strong developability liability. The options are either to perform a significant re-engineering (e.g. computational design of multiple mutations, perhaps guided by structural modeling or deep mutational scans), which is time-consuming and may risk affinity/epitope changes, or to drop the molecule from the pipeline
omniab.com
azom.com
. Often there are enough alternatives in early discovery that one doesn’t need to salvage a particularly sticky antibody unless it has a uniquely valuable specificity. If pursuing rescue, it may involve generating many variants and could require altering the isoelectric point or removing large hydrophobic motifs.
This framework is summarized in a logical form as: \text{If } \Delta\lambda_{\max} < 5~\text{nm and $k_D > 0$: \ Advance.} \] \[ \text{If } 5 \le \Delta\lambda_{\max} \le 10~\text{nm or $k_D \approx 0$: \ Consider engineering patches \& rescreen.} \] \[ \text{If } \Delta\lambda_{\max} > 10~\text{nm or $k_D < 0$: \ High risk – deprioritize or major re-engineering.} This rule-set (with the 5 nm cutoff in particular) is supported by published analyses of clinical-stage antibodies
omniab.com
omniab.com
 and technical reports from instrument providers linking k_D sign to stability
azom.com
. It aligns with the idea that truly “drug-like” antibodies very rarely show strong self-association signals early on
omniab.com
. Of course, real-world decisions also weigh other data (binding affinity, in vivo model results, etc.), but developability is increasingly viewed with equal importance to efficacy. The exact numeric criteria might be adjusted project by project, but having a formal threshold-based approach helps avoid the temptation to push forward a risky molecule based on hope or bias. It introduces an objective layer to the selection process, improving the probability of long-term success.
Example High-Throughput Protocol (PBS pH 7.4)
Below is a minimal protocol outline for performing AC-SINS in a standard condition (PBS pH 7.4) as a screening assay:
Prepare Immunogold Conjugate: Start with monodisperse ~20 nm gold nanoparticles. Mix with an optimized amount of goat anti-human Fc polyclonal antibody (enough to nearly saturate the AuNP surface). Incubate to allow passive adsorption of the capture antibody
link.springer.com
. After coating, add a capping agent such as PEG-thiol (e.g. PEG<sub>2000</sub>-SH) at ~0.5–1 mg/mL and incubate to block any remaining bare patches on the gold
academic.oup.com
. This PEGylation improves colloidal stability, especially if any buffer variations occur. Spin down or filter the conjugate to remove any aggregates and resuspend in PBS. Verify the conjugate’s baseline absorbance peak (~525 nm is expected; a small red shift from ~520 indicates successful antibody coating)
researchgate.net
. The conjugate can be stored short-term at 4 °C with a preservative.
Set Up Assay Plate: Use a clear flat-bottom 96-well plate (or 384-well for higher throughput). Include wells for (a) conjugate + plain PBS (no mAb) – to measure baseline λ_max, (b) conjugate + reference “low self-association” mAb (e.g. trastuzumab at 10 µg/mL), (c) conjugate + reference “high self-association” mAb (e.g. bococizumab or an in-house positive control, 10 µg/mL), and (d) each test mAb at the desired concentration (commonly 25 µg/mL, but a range like 5–50 µg/mL can be used to see any concentration trend). It’s wise to run each sample in duplicate wells for reliability. Add the mAb solutions (in PBS) to the plate first. Then add an equal volume of the prepared immunogold conjugate to each well (typical final AuNP absorbance ~0.5 AU at 520 nm in each well). Mix briefly by pipetting up and down or gentle shaking.
Incubation: Let the plate incubate at room temperature for a fixed time (e.g. 1 hour). Many interactions are rapid, and a shift can often be seen within minutes, but to ensure equilibrium and reproducibility, 1 hour is a safe choice. Avoid disturbing the plate (no shaking) during this time to allow interactions to occur undisturbed. Longer incubations (e.g. overnight) generally do not dramatically increase shifts for reversible self-association, but one should validate the kinetics – some very weak interactions might show a slow gradual shift increase. In high-throughput mode, 1 hour per plate is a common compromise.
Spectral Reading: Using a UV–Vis plate reader capable of spectral scans, measure the absorbance from ~450 to 750 nm for each well. Identify the wavelength of maximum absorbance for the conjugate peak for each sample. For well-behaved samples, this will be a single peak. Highly aggregating samples might yield an ill-defined broad peak; if so, note that (it may indicate precipitation). Most plate readers will report λ_max automatically. Alternatively, one can export the spectra and determine λ_max in analysis software.
Calculate Δλ_max: For each mAb sample, compute the shift Δλ_max = (sample λ_max) – (λ_max of the no-mAb control in PBS). Using the no-mAb conjugate well as the reference accounts for any slight day-to-day or plate-to-plate baseline differences. Likewise, one can look at the trastuzumab control: it should ideally show a near-zero shift (within ~1–2 nm of reference). The high-stick control (bococizumab, etc.) might show, for example, a 10–15 nm shift (depending on the chosen positive control’s properties)
pmc.ncbi.nlm.nih.gov
. If the controls behave unexpectedly (e.g. the supposed low-stick control gives a large shift), the data from that plate should be treated with caution or the assay repeated after troubleshooting.
Analysis: Collate the Δλ_max values for all samples. Plotting these in a bar graph can quickly highlight the outliers. Sometimes it’s useful to normalize the signals between the low and high controls for visualization (100% = shift of high control, 0% = shift of low control)
pmc.ncbi.nlm.nih.gov
, but for decision-making the absolute shifts in nm are typically used. Identify any antibodies above the pre-set risk thresholds (e.g. >5 nm or >10 nm). Those are flagged for further action (either elimination or engineering). Antibodies with shifts near zero are strong candidates to move forward, but ideally cross-validate their behavior with another method (for example, measure their k_D or run a CIC assay) to avoid any false negatives.
Following this protocol, Liu et al. (2014) screened dozens of antibodies in early discovery and were able to rapidly filter out a significant fraction that had high self-association, which correlated with poor developability
omniab.com
omniab.com
. By including orthogonal confirmation (like a DLS-based k_D measurement for hits), one can be confident that antibodies progressing to expensive development stages have a low risk of surprises related to colloidal stability.
Notes on Screening in Discovery Campaigns
In practice, AC-SINS (especially in PBS or another single condition) is used as part of a multi-parameter developability panel during lead selection
researchgate.net
researchgate.net
. A common paradigm in many companies is to run a battery of small-scale assays covering different liabilities – e.g. AC-SINS for self-association, CIC or ELISA-based polyspecificity assays for cross-reactivity, thermal stability (T<sub>m</sub>) measurements for unfolding propensity, perhaps proteolytic stability or expression titer for manufacturability, etc
nature.com
nature.com
. Each assay provides a piece of the puzzle, and together they form a “developability profile” for each antibody. AC-SINS in PBS is an excellent high-throughput triage tool to eliminate the worst offenders early, under easy conditions
researchgate.net
researchgate.net
. For example, in a campaign yielding 100 candidate mAbs, one might quickly find the ~20 with Δλ_max >10 nm and drop them, focusing on the remaining. In the next round, those remaining might be tested in a formulation-specific PS-SINS (say pH 6 histidine for a subcutaneous product) to further refine the ranking
researchgate.net
researchgate.net
. At that stage, one might integrate the data with other metrics like k_D and PEG C<sub>1/2</sub> to pick the top 5 antibodies that have the best overall developability profiles for more in-depth characterization
researchgate.net
researchgate.net
. It’s also worth noting that different projects may tailor the assays to their needs. For instance, if the target indication absolutely requires a high concentration (200 mg/mL) formulation in acetate buffer, the team might emphasize running AC-SINS in 10 mM acetate pH 5.5 (using a stabilized protocol) from the start, rather than PBS
researchgate.net
. Alternatively, they might run a “stress test” panel where each antibody is put through multiple conditions (PBS, histidine, with and without salt, etc.) to see if any are particularly sensitive to changes. Antibodies that show robustness (low self-association in all tested conditions) are obviously prime candidates. Those that only behave well in some conditions might be more challenging to formulate robustly. In summary, AC-SINS – especially the PBS version – is often the workhorse initial screen for colloidal stability in antibody discovery. Newer adaptations (PS-SINS, CS-SINS, etc.) allow those screenings to be extended to more relevant conditions when needed
researchgate.net
researchgate.net
. By combining the AC-SINS results with other developability signals (DLS, CIC, PEG, etc.), teams can make informed choices about which antibody to push forward. This up-front investment in developability screening greatly reduces the risk of late-stage failures due to insolubility or stability issues, ultimately saving time and cost in the development pipeline
nature.com
nature.com
.
Computational Modeling & Machine Learning in Developability
Alongside experimental assays, in silico approaches have rapidly evolved to predict and even mitigate antibody self-association and viscosity issues. These methods range from physics-based molecular simulations to data-driven machine learning models, and they play an increasingly important role in early antibody engineering
nature.com
nature.com
.
Molecular Simulation of Antibody Interactions
Molecular modeling can provide atomic-level insight into why an antibody tends to self-associate. For example, all-atom molecular dynamics (MD) simulations of concentrated antibody solutions have been used to probe intermolecular interactions. In these simulations, multiple antibody molecules (or fragments like Fabs) are placed in a periodic box at high concentration, and their behavior is observed over time. Such MD studies can calculate properties like the osmotic second virial coefficient B<sub>22</sub> or even directly compute the solution’s viscosity via the simulation of a shear flow
pubs.acs.org
. A recent study demonstrated the feasibility of predicting viscosity through long MD simulations: by simulating antibodies at ~100 mg/mL, the researchers could distinguish a high-viscosity antibody from a low-viscosity one based on the formation of transient clusters in the simulation
pubs.acs.org
pubs.acs.org
. While these fully detailed simulations are computationally expensive (often requiring supercomputers and many microseconds of sampling), they offer a virtual testing ground for understanding the molecular basis of self-association. Insights from MD – such as identifying a particular pair of CDR residues frequently sticking together in different molecules – can inform mutation strategies to disrupt those contacts. Another application of simulation is the use of coarse-grained models or statistical mechanics to estimate protein-protein interaction parameters. For instance, one can approximate antibodies as charged spheres or ellipsoids with patchy interactions and use Monte Carlo simulations to estimate how changes in charge or hydrophobicity would affect k_D or B<sub>22</sub>. These simplified models capture trends (like “increasing net charge will likely increase the repulsion and thus B<sub>22</sub>”) without the need to simulate every atom in the antibody
unchainedlabs.com
technologynetworks.com
. They have been used to rationalize experimental findings, such as why certain mutations dramatically improve colloidal stability – often the mutation removes a “sticky” patch or adds a charge that enhances electrostatic repulsion, as the simulations confirm. Molecular docking is another computational technique in this space: one can dock an antibody structure onto itself (or onto copies of itself) to predict potential self-binding interfaces. Software can search for complementary surfaces on two antibody Fab structures that might bind if they come into contact. High-scoring docked configurations often correspond to actual aggregation pathways. For example, if a particular hydrophobic patch on one antibody tends to dock onto a groove on another antibody in many simulation snapshots, this suggests a mode of self-association that could be happening in solution. Such information again guides engineering – perhaps mutating a key hydrophobic residue in that patch could disrupt the docked complex. It should be noted that in silico predictions of self-association are challenging due to the size and flexibility of antibodies and the often subtle, multivalent nature of their interactions. Simulations must capture a delicate balance of forces (hydrophobic, electrostatic, steric) in an accurate forcefield. Recent advances in GPU-accelerated computing and enhanced sampling algorithms are making these studies more tractable, but they are still not routine for every project.
Machine Learning Models for Developability
The scarcity of large experimental datasets for antibody developability (each antibody requires significant effort to characterize) initially limited the use of machine learning (ML). However, in recent years, researchers have begun compiling datasets of antibodies with known viscosity or aggregation outcomes, and ML techniques are being applied to predict properties from sequence or structure
nature.com
nature.com
. One example is an interpretable deep learning model developed by researchers at Pfizer, called PfAbNet-viscosity. This model uses a 3D convolutional neural network that takes as input the electrostatic potential map of the antibody’s Fv region (essentially a 3D grid representing the charge distribution on the antibody surface)
nature.com
nature.com
. Trained on a relatively small dataset (tens of antibodies with measured high-concentration viscosities), PfAbNet was able to generalize and predict viscosity categories with notable accuracy
nature.com
nature.com
. The clever aspect is the use of a “biophysically meaningful representation” – by feeding the network the physics-based electrostatic map, it helps the model learn features like “a large positive patch here is bad” without needing thousands of examples
nature.com
. Indeed, feature attribution analysis in that work showed the model focused on regions corresponding to known sticky patches (e.g. CDR loops with concentrated positive charge)
nature.com
nature.com
. This demonstrates that ML can capture key drivers of viscosity even with limited data, by incorporating domain knowledge (here, the importance of electrostatics). Another example comes from a collaboration between academia and industry (Lai et al. 2022), where researchers measured viscosity and aggregation rates for a set of 20 antibodies and then built ML models using input features derived from both sequence and molecular dynamics simulations
pubmed.ncbi.nlm.nih.gov
pubmed.ncbi.nlm.nih.gov
. They found that a simple k-nearest neighbors regression model could predict aggregation rates with high correlation (r ~0.89) using just two features: (1) the positive charge cluster on one of the heavy chain CDRs (specifically, a spatial positive charge map over H2) and (2) the solvent-accessible hydrophobic surface area on the Fv
pubmed.ncbi.nlm.nih.gov
. In other words, antibodies that had a strongly positive H2 patch and a large exposed hydrophobic area tended to aggregate fastest, and the ML picked up on that
pubmed.ncbi.nlm.nih.gov
pubmed.ncbi.nlm.nih.gov
. For viscosity classification (high vs low), a logistic regression model using the net negative charge on heavy and light chain variable regions performed best (antibodies with more negative charges in their Fv were less likely to be high viscosity)
pubmed.ncbi.nlm.nih.gov
pubmed.ncbi.nlm.nih.gov
. These models, while relatively simple, quantitatively confirmed intuitive rules: “hydrophobic + positively-charged = problematic” and “more negative charge = beneficial” in the context of self-association. Beyond these specific examples, the field has seen a proliferation of ML approaches: random forest models using sequence-derived descriptors (like patches calculated by software tools), support vector machines trained on experimental developability indices, and even language model embeddings of antibody sequences (like using transformer neural networks to generate features for each antibody, then training a predictor on those)
biorxiv.org
pubs.acs.org
. One group used a transfer learning model (ProtT5, a protein language model) combined with a random forest to predict viscosity from sequence alone, and reported good performance on distinguishing high-viscosity candidates
sciencedirect.com
sciencedirect.com
. Another interesting application is using ML to propose mutations to fix a problematic antibody. In a 2023 study, an interpretable ML model was trained on known viscosity outcomes of many variants of an antibody and was able to highlight which residues contribute most to high viscosity; by altering those residues (either to different amino acids suggested by the model or via computational design), they produced variants with significantly reduced viscosity
tandfonline.com
biorxiv.org
. One should keep in mind that ML models are only as good as the data they train on. Many early models trained on small datasets of mostly clinical or late-stage antibodies, which are a biased set (they mostly consist of already developable antibodies, since problematic ones often never reach late stage)
pubmed.ncbi.nlm.nih.gov
pubmed.ncbi.nlm.nih.gov
. This can limit the model’s ability to recognize truly bad actors – essentially, the training data didn’t contain the “failures” because those were weeded out. To combat this, efforts are underway to share developability data (e.g. through the Therapeutic Data Commons or other initiatives) and to include negative examples from internal discovery programs. In practice, computational predictions are being used in tandem with experiments. A likely workflow might be: as soon as an antibody sequence is obtained (from panning or B-cell cloning), run a quick in silico developability screen – calculate its hydrophobic patches, net charges, perhaps run it through a ML model like the one by Sormanni et al. for solubility
nature.com
nature.com
. Sormanni’s CamSol approach, for instance, can assign an intrinsic solubility score to a sequence without any lab work
nature.com
nature.com
. If a candidate has a poor score, one might deprioritize it or immediately consider engineering even before lab testing
nature.com
nature.com
. Meanwhile, the top candidates go into wet-lab assays (like AC-SINS, DLS, etc.) for confirmation. This synergy can shorten the cycle – computational methods filter the thousands of initial possibilities down to a few hundred, which are then assayed experimentally, and the results of those assays can in turn feed back to refine the models. An exciting frontier is combining in silico and in vitro in closed-loop design. For example, one could envision an algorithm that designs mutations to maximize a computational “developability score” while preserving the antibody’s binding, then those designs are tested experimentally to validate and further improve the algorithm. AstraZeneca’s recent work using HDX-MS plus in silico modeling to rationally design lower-viscosity mutants is a case in point: they used experimental data to guide a computational search for better variants, then used high-throughput screens (DLS, AC-SINS) to evaluate ~70 designed variants, from which several showed dramatically improved viscosity
pubmed.ncbi.nlm.nih.gov
pubmed.ncbi.nlm.nih.gov
. These successful outcomes can then become new training data for ML models, creating a virtuous cycle of improvement. In conclusion, computational methods – from detailed MD simulations to quick sequence-based ML predictions – are now important tools in the antibody developability toolkit. They help identify risk factors (like certain motifs or surface patterns) even before an antibody is made, and they guide the engineering of better molecules. While they won’t fully replace empirical testing (given the complexity of antibody behavior), they markedly enhance our ability to understand and optimize antibodies in a resource- and time-efficient manner
nature.com
nature.com
.
Formulation & Manufacturing Considerations for High-Concentration mAbs
Even after selecting an antibody with low self-interaction propensity, a formulation scientist’s task is to create a stable, patient-friendly drug product – typically a high-concentration liquid that remains clear, non-viscous, and stable over shelf-life. Key considerations include the choice of buffer, pH, excipients (stabilizers, surfactants, etc.), and the manufacturability of the formulation (how it behaves in filtration, filling, etc.). Early developability assessments like AC-SINS and k_D inform these choices, but additional experimentation is needed to finalize the formulation.
Buffer and pH Selection
Monoclonal antibodies are generally formulated in mild aqueous buffers, often at pH in the range of 5.0 to 6.5. A systematic review of approved high-concentration antibody products (HCAPs) showed that the vast majority use a pH between 5 and 6
pmc.ncbi.nlm.nih.gov
pmc.ncbi.nlm.nih.gov
. Only a handful of products were as low as pH 4.5 or as high as pH 7.0; most clustered around pH 5.5–6.0
pmc.ncbi.nlm.nih.gov
pmc.ncbi.nlm.nih.gov
. There are multiple reasons for this. First, antibodies often have an isoelectric point (pI) in the 7–9 range; formulating a bit below the pI (at pH 5–6) gives them a net positive charge, which can improve colloidal stability by ensuring like-charge repulsion (provided the antibody doesn’t have an internal polynegative patch). Second, many chemical degradation pathways (like asparagine deamidation or aspartate isomerization) are minimized in the slightly acidic pH range
pmc.ncbi.nlm.nih.gov
pmc.ncbi.nlm.nih.gov
. At higher pH, you risk faster deamidation; at lower pH, you might trigger aggregation or denaturation for some IgGs. Thus, pH ~5.5 is a sweet spot balancing physical and chemical stability for many antibodies
pmc.ncbi.nlm.nih.gov
pmc.ncbi.nlm.nih.gov
. Common buffers that maintain pH in this range include histidine (widely used around pH 5.5–6.0), acetate (pH 5.0), citrate (pH 6.5 or lower), and occasionally phosphate (pH 6–7) or succinate. Histidine, in particular, is popular for high-concentration formulations: it has good buffering capacity around pH 5.5–6.0 and is relatively benign (it doesn’t induce heavy metal catalysis or other side reactions). Moreover, histidine buffers exhibit minimal pH shift upon freezing/thawing, which is useful if the product might be frozen during manufacturing
pmc.ncbi.nlm.nih.gov
pmc.ncbi.nlm.nih.gov
. The review of marketed products found histidine in a large fraction of SC formulations, often in combination with other excipients
pmc.ncbi.nlm.nih.gov
pmc.ncbi.nlm.nih.gov
.
Excipients to Mitigate Viscosity and Instability
High-concentration antibodies can suffer from issues like aggregation, phase separation, or high viscosity. Formulation scientists employ various excipients to counteract these. According to the same HCAP review, the typical excipient “classes” and their roles are
pmc.ncbi.nlm.nih.gov
pmc.ncbi.nlm.nih.gov
:
Tonicity agents: e.g. sucrose, trehalose, glycerol, mannitol, or simple salts like NaCl to ensure the formulation is isotonic. These mostly balance osmotic pressure but can have secondary stabilizing effects. Sucrose is most common, present in a majority of products, as it also acts as a stabilizer via preferential hydration
pmc.ncbi.nlm.nih.gov
pmc.ncbi.nlm.nih.gov
.
Surfactants: e.g. Polysorbate 20 or 80 (or poloxamers in a couple of cases). These protect the antibody from interfacial stress (like agitation or surface adsorption) and prevent aggregation caused by surface denaturation
pmc.ncbi.nlm.nih.gov
. They don’t directly affect viscosity in bulk, but ensure that handling (shaking, filling) doesn’t create particulates.
Amino acid additives: Notably L-arginine, and to a lesser extent L-proline or glycine, are frequently used. Arginine stands out as a multi-functional excipient: it can act as a viscosity reducer, stabilizer, and even as a tonicity agent at high concentrations
pmc.ncbi.nlm.nih.gov
pmc.ncbi.nlm.nih.gov
. Arginine (often as arginine hydrochloride) has a chaotropic effect that can disrupt protein-protein attractions – effectively “shielding” sticky patches through preferential binding or by salting-in effects. Many modern formulations include arginine to keep viscosity low; in fact, arginine was present in several products that lacked sugars or salt for tonicity, indicating its dual role
pmc.ncbi.nlm.nih.gov
pmc.ncbi.nlm.nih.gov
. Proline and glycine can also serve as stabilizers/viscosity modifiers (and as tonicity agents). Glycine, for example, is used in some subcutaneous IG products at very high levels (it also acts as a buffer in those cases)
pmc.ncbi.nlm.nih.gov
pmc.ncbi.nlm.nih.gov
.
Salt: Moderate levels of salt (e.g. NaCl) are sometimes included specifically to reduce viscosity
pmc.ncbi.nlm.nih.gov
pmc.ncbi.nlm.nih.gov
. Low ionic strength can lead to strong electrostatic repulsion between antibodies (which is good for preventing self-association in some cases, but too much repulsion can promote structures like “string-of-beads” that paradoxically increase viscosity due to alignment of molecules). By adding a bit of salt, one can screen just enough charge to allow molecules to pack more freely and reduce viscous drag
patents.google.com
patents.google.com
. NaCl is present in a few formulations at ionic strengths that suggest its role is beyond tonicity (e.g. 50–100 mM added NaCl even when isotonicity was already achieved with sugars)
pmc.ncbi.nlm.nih.gov
pmc.ncbi.nlm.nih.gov
. This likely was to address viscosity or aggregation issues during development.
Chelators and antioxidants: EDTA or DTPA might be added to chelate trace metals (preventing metal-catalyzed aggregation), and methionine can be added as an antioxidant scavenger. These don’t directly influence viscosity but improve stability.
“Viscosity modifier” category: In some literature, arginine, NaCl, glycine, and sometimes surfactants are explicitly called out for their viscosity-lowering effects
pmc.ncbi.nlm.nih.gov
pmc.ncbi.nlm.nih.gov
. Arginine is widely cited as the most effective general viscosity reducer for antibodies
pmc.ncbi.nlm.nih.gov
pmc.ncbi.nlm.nih.gov
. For example, one formulation of a highly viscous antibody was able to go from unusable to acceptable by including ~50 mM arginine HCl and ~5% sucrose – the arginine disrupted self-association while sucrose improved physical stability during freeze-thaw.
Excipients are chosen through an empirical formulation development process. Often a design of experiments (DoE) approach is used, where small-scale formulations are made exploring a matrix of pH levels, buffers, and additives, then stress tested (e.g. high temperature stability, viscosity at high conc, etc.). The outcome is selecting a combination that meets multiple criteria: low viscosity, high stability, acceptable injection volume/osmolality, etc.
Managing High Viscosity in Manufacturing
Even with an optimal formulation composition, the sheer fact that the protein is at 100–200 mg/mL brings manufacturing challenges. Viscosity affects practically every unit operation in drug product manufacturing
cytivalifesciences.com
cytivalifesciences.com
:
Filtration: Highly viscous solutions flow slowly through filters. Sterile filtration of a 200 mg/mL mAb can require much higher driving pressures or larger filter areas, which is costly and can stress the product
cytivalifesciences.com
cytivalifesciences.com
. Additionally, some filters may foul or not achieve the needed throughput with viscous feeds
cytivalifesciences.com
cytivalifesciences.com
. Manufacturers often have to select specific low-binding, high-flow filter membranes (and perform extensive validation) to ensure sterilizing filtration is feasible for a high-concentration formulation
cytivalifesciences.com
cytivalifesciences.com
.
Mixing and Filling: Thick solutions are harder to mix without introducing air (which can cause foaming or aggregation) and are harder to accurately dispense. Pump systems for fill-finish need calibration for high viscosity to avoid under/over-filling. At large scale, even bulk drug substance transfer is impacted – pumping a viscous solution through hoses can require higher pressures, potentially introducing shear stress. Investing in positive-displacement pumps or pressurized vessels might be needed to handle the fluid. Additionally, the residual hold-up volume in equipment becomes a bigger deal when the product is literally like syrup (losing 100 mL in tubing that can't be fully flushed might correspond to many doses of a high-value drug).
Ultrafiltration/diafiltration (UF/DF): The final step of concentrating the drug to the target strength can be time-consuming. For example, to reach >100 mg/mL, one might need to concentrate 10-fold or more. As concentration rises, filter membranes experience dramatically increased transmembrane pressure to maintain flow
cytivalifesciences.com
cytivalifesciences.com
. At some point, flow nearly stops. Process innovations like single-pass tangential flow filtration (SPTFF) are used – this can concentrate product in one go without recirculation, mitigating shear damage and avoiding long hold times in pumps
cytivalifesciences.com
cytivalifesciences.com
. By carefully controlling flow and pressure, SPTFF can achieve high final concentrations more gently than traditional TFF setups. Manufacturers also sometimes perform diafiltration (buffer exchange) at a lower concentration, then concentrate at the very end, to reduce time pumping at high viscosity
cytivalifesciences.com
. Another trick is to do concentration at cooler temperatures (since viscosity of liquids generally decreases at higher temperature – but protein stability may limit how warm you can go, so some do it cold to reduce aggregation kinetics).
Freezing/thawing and storage: Many bulk drug substances are stored frozen to lengthen shelf-life before final fill. Freezing highly concentrated protein solutions can cause cryoconcentration (local regions in the frozen matrix where protein and solutes become extremely concentrated as ice forms). Upon thawing, this can lead to aggregates if not properly controlled
cytivalifesciences.com
cytivalifesciences.com
. Controlled-rate plate freezers are often employed to freeze high-concentration solutions quickly and uniformly, preventing excessive concentration gradients
cytivalifesciences.com
cytivalifesciences.com
. Specialized freezing containers (bag systems that form thin layers) help ensure even freezing. The formulation might also include cryoprotectants (like the disaccharides mentioned) to mitigate freeze concentration effects
pmc.ncbi.nlm.nih.gov
pmc.ncbi.nlm.nih.gov
.
From a manufacturing perspective, each potential problem has a solution, but it adds complexity or cost. That’s why, again, keeping viscosity within a reasonable range (<20 cP) is a design target – beyond that, the number of workarounds needed starts to pile up (e.g. heating lines, custom devices, etc.). If an antibody absolutely cannot be made less viscous by formulation or mutation, one might consider alternative presentations: for example, a lyophilized cake that is reconstituted (so you avoid pumping a thick liquid, though the patient still ultimately injects it after reconstitution). Historically, many early mAbs were lyophilized for stability, but today, thanks to better molecules and formulations, >90% of new mAbs are launched as ready-to-use solutions
pmc.ncbi.nlm.nih.gov
pmc.ncbi.nlm.nih.gov
, which patients and providers vastly prefer.
Example: Arginine to the Rescue
A concrete case study in formulation adjustment is Arginine’s effect on viscosity. Wang et al. (Mol. Pharm. 2015) examined two IgG1 antibodies at high concentration and tested various amino acids and salts on their solution viscosity. They found that adding 50 mM arginine or lysine reduced viscosity by up to 30–50% in some cases
sciencedirect.com
pmc.ncbi.nlm.nih.gov
. The effect was attributed to arginine’s ability to interact with exposed aromatic and hydrophobic regions on the antibody, thereby preventing antibody-antibody contact (arginine is known to bind weakly to tryptophans and tyrosines on protein surfaces). NaCl also showed a modest viscosity-lowering effect, likely by screening electrostatic attractions
pmc.ncbi.nlm.nih.gov
. Interestingly, the combination of arginine plus a little NaCl was sometimes synergistic – arginine primarily mitigated hydrophobic interactions while salt handled charge interactions, together yielding a much more free-flowing solution. These findings have translated to real products: as noted, many approved formulations include arginine specifically for this reason
pmc.ncbi.nlm.nih.gov
pmc.ncbi.nlm.nih.gov
. Another interesting excipient is hyaluronidase, not as a formulation component of the antibody solution itself but co-administered to facilitate SC delivery (e.g. Halozyme’s ENHANZE technology). Hyaluronidase enzyme breaks down hyaluronan in subcutaneous tissue, temporarily reducing resistance and allowing larger volumes (and somewhat higher viscosity) to be injected more comfortably. For instance, a combination of hyaluronidase with certain mAbs has enabled 5 mL injections that would otherwise be infeasible. While this doesn’t change the antibody formulation per se, it’s part of the delivery consideration. If an antibody can only be formulated to, say, 150 mg/mL due to viscosity, but the dose needs 300 mg, one could either give two injections or use hyaluronidase to allow a single larger-volume injection. Several products (e.g. rituximab/hyaluronidase, trastuzumab/hyaluronidase) are on the market using this approach. It underscores that formulation design sometimes extends beyond the vial – to how you can help the patient receive the drug.
Integration with Developability Screening
How does all this tie back to the early screening assays like AC-SINS? Essentially, those early assays give clues about what formulation strategies will be needed. For example, if an antibody shows slight self-association in PBS but much less in 150 mM NaCl, one might infer that controlling electrostatics is key – so formulating with a moderate salt or at a pH near the antibody’s pI might help. If AC-SINS indicates strong hydrophobic-driven self-association (perhaps because adding arginine in the assay dramatically reduced the shift, implying hydrophobic contacts were involved), then one plans to include arginine or similar excipients in the formulation. Early on, one also flags whether the antibody might require non-standard measures (like the hyaluronidase co-formulation or a device for delivery). The interplay between formulation scientists and discovery scientists is growing: formulators sometimes participate in lead selection, bringing up issues like “Antibody A is excellent on potency but has a pI of 9 and high self-interaction, so if we choose it, we will likely need a very specific formulation (low pH, arginine, etc.) to make it work – are we prepared for that, or should we lean toward Antibody B which is slightly less potent but far easier to formulate?” These discussions ensure that by the time a lead is chosen, there’s already a roadmap for formulation, and hopefully no unsurmountable stability issues lie ahead. In summary, developability doesn’t end with selecting a low Δλ_max antibody – it continues into crafting the right formulation. The goal is a stable, manufacturable product: typically a liquid at pH ~5.5, isotonic, containing the mAb at high concentration along with excipients like sugars (for stability), polysorbate (for shear protection), and amino acids or salts (to modulate viscosity)
pmc.ncbi.nlm.nih.gov
pmc.ncbi.nlm.nih.gov
. Each antibody may need a slightly different recipe, but thanks to early screening, the formulation team knows where the challenges lie (if any) – whether it’s a tendency to self-associate (counter with arginine), a high opalescence risk (maybe add a surfactant or adjust pH), or potential for aggregation (use stabilizing sugars and avoid extreme pH or temperature in processing). The outcome of a successful collaboration between discovery and formulation is a drug product that the patient can receive as a convenient injection, which remains safe and effective throughout its shelf life.
References
Liu et al., 2014. High-throughput screening for developability during early-stage antibody discovery using self-interaction nanoparticle spectroscopy (AC-SINS). mAbs 6(2): 483–492. PMID: 24492294. PMCID: PMC3984336. 
researchgate.net
pmc.ncbi.nlm.nih.gov
Phan et al., 2022. High-throughput profiling of antibody self-association in multiple formulation conditions by PEG-stabilized SINS. mAbs 14(1): 2094750. PMCID: PMC9291693. 
researchgate.net
researchgate.net
Geng et al., 2016 (Mol. Pharm.). Measurements of monoclonal antibody self-association are correlated with complex biophysical properties. Molecular Pharmaceutics 13(5): 1636–1645. DOI: 10.1021/acs.molpharmaceut.6b00071. 
researchgate.net
pmc.ncbi.nlm.nih.gov
Geng et al., 2016 (Bioconj. Chem.). Facile preparation of stable antibody–gold conjugates and application to AC-SINS. Bioconjugate Chemistry 27(10): 2287–2300. PMID: 27494306. 
academic.oup.com
link.springer.com
Wu et al., 2015. Discovery of highly soluble antibodies prior to purification using AC-SINS. Protein Engineering, Design & Selection 28(10): 403–414. DOI: 10.1093/protein/gzv045. 
academic.oup.com
tandfonline.com
Avery et al., 2018. Establishing in vitro–in vivo correlations to screen mAbs for human PK properties. mAbs 10(2): 244–255. PMCID: PMC5825195. 
omniab.com
omniab.com
Jain et al., 2017. Biophysical properties of the clinical-stage antibody landscape. PNAS 114(5): 944–949. DOI: 10.1073/pnas.1616408114. 
pnas.org
pnas.org
Zarzar et al., 2023. High concentration formulation developability approaches and considerations. mAbs 15(1): 2211185. PMCID: PMC10190182. 
researchgate.net
researchgate.net
Bailly et al., 2020. Predicting antibody developability profiles through early-stage discovery screening. mAbs 12(1): 1743053. PMCID: PMC7153844. 
pmc.ncbi.nlm.nih.gov
pmc.ncbi.nlm.nih.gov
Hedberg et al., 2018. Cross-interaction chromatography as a rapid screening technique for predictability of antibody behavior at high concentration. J. Chromatogr. B 1095: 164–176. DOI: 10.1016/j.jchromb.2018.07.005. 
pmc.ncbi.nlm.nih.gov
Kelly et al., 2015. High-throughput cross-interaction chromatography: screening mAbs for polyspecificity and self-association. Methods in Molecular Biology 1313: 23–35. (MIT Open Access Article) 
pmc.ncbi.nlm.nih.gov
sciencedirect.com
Wyatt Technology, 2016. The diffusion interaction parameter k<sub>D</sub> as an indicator of colloidal and thermal stability (Application Note WP5004). 
azom.com
Oeller et al., 2021. Open-source automated PEG precipitation assay to assess relative solubility of therapeutic antibodies. Communications Biology 4: 1210. PMCID: PMC8578320. 
nature.com
www-vendruscolo.ch.cam.ac.uk
Sormanni et al., 2017. Rapid and accurate in silico solubility screening of a monoclonal antibody library. Scientific Reports 7: 8200. PMID: 28894102. 
nature.com
nature.com
Wälchli et al., 2020. Relationship of PEG-induced precipitation to protein–protein interactions and aggregation propensity at high concentration. mAbs 12(1): 1694764. DOI: 10.1080/19420862.2019.1694764. 
researchgate.net
Bhandari et al., 2023. Prediction of antibody viscosity from dilute solution measurements. Antibodies 12(4): 78. DOI: 10.3390/antib12040078. 
mdpi.com
mdpi.com
Lefèvre et al., 2025. Enhanced rational protein engineering to reduce viscosity in high-concentration IgG1 antibody solutions. mAbs 17(1): 2543771. Epub 2025 Aug 8. DOI: 10.1080/19420862.2025.2543771. 
pubmed.ncbi.nlm.nih.gov
pubmed.ncbi.nlm.nih.gov
Ferrara et al., 2022. Pandemic-enabled comparison of discovery platforms reveals trade-offs in antibody screening and selection. Nature Communications 13: 104. DOI: 10.1038/s41467-021-27614-1. 
tandfonline.com
Estep et al., 2015. An alternative assay to hydrophobic interaction chromatography for ranking non-specific protein interactions of antibodies. mAbs 7(3): 558–571. DOI: 10.1080/19420862.2015.1029211. 
researchgate.net
Yadav et al., 2018. Assay for determining the potential to self-associate of a protein using concentration-dependent self-interaction nanoparticle spectroscopy (Patent WO2018035470A1). 
patents.google.com
patents.google.com
van de Lavoir et al., 2024. Heavy chain-only antibodies with a stabilized human V<sub>H</sub> in transgenic chickens for therapeutic antibody discovery (White Paper/Preprint). 
omniab.com
omniab.com
Lai et al., 2022. Machine learning prediction of antibody aggregation and viscosity for high concentration formulation development. mAbs 14(1): 2026208. PMCID: PMC8794240. 
pubmed.ncbi.nlm.nih.gov
pubmed.ncbi.nlm.nih.gov
Rai et al., 2023. Low-data interpretable deep learning prediction of antibody viscosity using a biophysically meaningful representation. Scientific Reports 13: 2917. DOI: 10.1038/s41598-023-28841-4. 
nature.com
nature.com
Clemens et al., 2023. A systematic review of commercial high concentration antibody products: formulation composition, design and processing considerations. European Journal of Pharmaceutics and Biopharmaceutics 187: 50–71. PMCID: PMC10228404. 
pmc.ncbi.nlm.nih.gov
pmc.ncbi.nlm.nih.gov
Wang et al., 2015. Viscosity-lowering effect of amino acids and salts on highly concentrated solutions of two IgG1 monoclonal antibodies. Molecular Pharmaceutics 12(12): 4478–4487. DOI: 10.1021/acs.molpharmaceut.5b00511. 
sciencedirect.com
pmc.ncbi.nlm.nih.gov
Cytiva, 2021. Manufacturing challenges with high concentration biologics (Online article). 
cytivalifesciences.com
cytivalifesciences.com
Citations

High-throughput screening for developability during early-stage ...

https://pmc.ncbi.nlm.nih.gov/articles/PMC3984336/

Ultradilute Measurements of Self-Association for the Identification of Antibodies with Favorable High-Concentration Solution Properties | Request PDF

https://www.researchgate.net/publication/352267491_Ultradilute_Measurements_of_Self-Association_for_the_Identification_of_Antibodies_with_Favorable_High-Concentration_Solution_Properties
WO2018035470A1 - Assay for determining potential to self-association of a protein using concentration-dependent self-interaction nanoparticle spectroscopy - Google Patents

https://patents.google.com/patent/WO2018035470A1/en
WO2018035470A1 - Assay for determining potential to self-association of a protein using concentration-dependent self-interaction nanoparticle spectroscopy - Google Patents

https://patents.google.com/patent/WO2018035470A1/en

Heavy chain-only antibodies with a stabilized human VH in transgenic chickens for therapeutic antibo

https://www.omniab.com/wp-content/uploads/2025/01/Heavy-chain-only-antibodies-with-a-stabilized-human-VH-in-transgenic-chickens-for-therapeutic-antibody-discovery-1.pdf

Ultradilute Measurements of Self-Association for the Identification of Antibodies with Favorable High-Concentration Solution Properties | Request PDF

https://www.researchgate.net/publication/352267491_Ultradilute_Measurements_of_Self-Association_for_the_Identification_of_Antibodies_with_Favorable_High-Concentration_Solution_Properties

Ultradilute Measurements of Self-Association for the Identification of Antibodies with Favorable High-Concentration Solution Properties | Request PDF

https://www.researchgate.net/publication/352267491_Ultradilute_Measurements_of_Self-Association_for_the_Identification_of_Antibodies_with_Favorable_High-Concentration_Solution_Properties

Heavy chain-only antibodies with a stabilized human VH in transgenic chickens for therapeutic antibo

https://www.omniab.com/wp-content/uploads/2025/01/Heavy-chain-only-antibodies-with-a-stabilized-human-VH-in-transgenic-chickens-for-therapeutic-antibody-discovery-1.pdf

Ultradilute Measurements of Self-Association for the Identification of Antibodies with Favorable High-Concentration Solution Properties | Request PDF

https://www.researchgate.net/publication/352267491_Ultradilute_Measurements_of_Self-Association_for_the_Identification_of_Antibodies_with_Favorable_High-Concentration_Solution_Properties

Application of Affinity-Capture Self-Interaction Nanoparticle ...

https://link.springer.com/article/10.1007/s11095-021-03026-8

Discovery of highly soluble antibodies prior to purification using ...

https://academic.oup.com/peds/article/28/10/403/1477975
WO2018035470A1 - Assay for determining potential to self-association of a protein using concentration-dependent self-interaction nanoparticle spectroscopy - Google Patents

https://patents.google.com/patent/WO2018035470A1/en
WO2018035470A1 - Assay for determining potential to self-association of a protein using concentration-dependent self-interaction nanoparticle spectroscopy - Google Patents

https://patents.google.com/patent/WO2018035470A1/en

Reduction of therapeutic antibody self-association using yeast ...

https://pmc.ncbi.nlm.nih.gov/articles/PMC9704398/

Heavy chain-only antibodies with a stabilized human VH in transgenic chickens for therapeutic antibo

https://www.omniab.com/wp-content/uploads/2025/01/Heavy-chain-only-antibodies-with-a-stabilized-human-VH-in-transgenic-chickens-for-therapeutic-antibody-discovery-1.pdf

Heavy chain-only antibodies with a stabilized human VH in transgenic chickens for therapeutic antibo

https://www.omniab.com/wp-content/uploads/2025/01/Heavy-chain-only-antibodies-with-a-stabilized-human-VH-in-transgenic-chickens-for-therapeutic-antibody-discovery-1.pdf
Reduction of therapeutic antibody self-association using yeast ...

https://www.tandfonline.com/doi/full/10.1080/19420862.2022.2146629

High-throughput profiling of antibody self-association in multiple ...

https://www.researchgate.net/publication/361979099_High-throughput_profiling_of_antibody_self-association_in_multiple_formulation_conditions_by_PEG_stabilized_self-interaction_nanoparticle_spectroscopy
WO2018035470A1 - Assay for determining potential to self-association of a protein using concentration-dependent self-interaction nanoparticle spectroscopy - Google Patents

https://patents.google.com/patent/WO2018035470A1/en
WO2018035470A1 - Assay for determining potential to self-association of a protein using concentration-dependent self-interaction nanoparticle spectroscopy - Google Patents

https://patents.google.com/patent/WO2018035470A1/en
WO2018035470A1 - Assay for determining potential to self-association of a protein using concentration-dependent self-interaction nanoparticle spectroscopy - Google Patents

https://patents.google.com/patent/WO2018035470A1/en
WO2018035470A1 - Assay for determining potential to self-association of a protein using concentration-dependent self-interaction nanoparticle spectroscopy - Google Patents

https://patents.google.com/patent/WO2018035470A1/en
WO2018035470A1 - Assay for determining potential to self-association of a protein using concentration-dependent self-interaction nanoparticle spectroscopy - Google Patents

https://patents.google.com/patent/WO2018035470A1/en

Ultradilute Measurements of Self-Association for the Identification of Antibodies with Favorable High-Concentration Solution Properties | Request PDF

https://www.researchgate.net/publication/352267491_Ultradilute_Measurements_of_Self-Association_for_the_Identification_of_Antibodies_with_Favorable_High-Concentration_Solution_Properties

Ultradilute Measurements of Self-Association for the Identification of Antibodies with Favorable High-Concentration Solution Properties | Request PDF

https://www.researchgate.net/publication/352267491_Ultradilute_Measurements_of_Self-Association_for_the_Identification_of_Antibodies_with_Favorable_High-Concentration_Solution_Properties

Ultradilute Measurements of Self-Association for the Identification of Antibodies with Favorable High-Concentration Solution Properties | Request PDF

https://www.researchgate.net/publication/352267491_Ultradilute_Measurements_of_Self-Association_for_the_Identification_of_Antibodies_with_Favorable_High-Concentration_Solution_Properties

Ultradilute Measurements of Self-Association for the Identification of Antibodies with Favorable High-Concentration Solution Properties | Request PDF

https://www.researchgate.net/publication/352267491_Ultradilute_Measurements_of_Self-Association_for_the_Identification_of_Antibodies_with_Favorable_High-Concentration_Solution_Properties

[PDF] Spot aggregation early with B and k on Stunner - Unchained Labs

https://www.unchainedlabs.com/wp-content/uploads/2021/11/AN-Spot-aggregation-early-with-B22-and-kD-on-Stunner.pdf

[PDF] Application Note | Technology Networks

https://www.technologynetworks.com/vaccines/go/lc/view-application-note-231333

The Diffusion Interaction Parameter (Kd)

https://www.azom.com/article.aspx?ArticleID=13108

Machine learning prediction of antibody aggregation and viscosity for high concentration formulation development of protein therapeutics - PubMed

https://pubmed.ncbi.nlm.nih.gov/35075980/

Machine learning prediction of antibody aggregation and viscosity for high concentration formulation development of protein therapeutics - PubMed

https://pubmed.ncbi.nlm.nih.gov/35075980/

Prediction of Antibody Viscosity from Dilute Solution Measurements

https://www.mdpi.com/2073-4468/12/4/78

Machine learning prediction of antibody aggregation and viscosity for high concentration formulation development of protein therapeutics - PubMed

https://pubmed.ncbi.nlm.nih.gov/35075980/

High throughput cross-interaction measures for human IgG1 ...

https://pmc.ncbi.nlm.nih.gov/articles/PMC4622737/

Rapid and accurate in silico solubility screening of a monoclonal ...

https://pubmed.ncbi.nlm.nih.gov/28811609/

In vitro and in silico assessment of the developability of a designed ...

https://www.researchgate.net/publication/329486000_In_vitro_and_in_silico_assessment_of_the_developability_of_a_designed_monoclonal_antibody_library

Rapid and accurate in silico solubility screening of a monoclonal ...

https://www.nature.com/articles/s41598-017-07800-w
The CamSol method for protein solubility prediction - Vendruscolo Lab

https://www-vendruscolo.ch.cam.ac.uk/camsolmethod.html

Ultra-high concentration low-viscosity subcutaneous antibody ...

https://www.sciencedirect.com/science/article/pii/S0168365925009083

Discovery of compounds with viscosity-reducing effects on ...

https://www.sciencedirect.com/science/article/pii/S2001037022004408

Ultradilute Measurements of Self-Association for the Identification of Antibodies with Favorable High-Concentration Solution Properties | Request PDF

https://www.researchgate.net/publication/352267491_Ultradilute_Measurements_of_Self-Association_for_the_Identification_of_Antibodies_with_Favorable_High-Concentration_Solution_Properties

Ultradilute Measurements of Self-Association for the Identification of Antibodies with Favorable High-Concentration Solution Properties | Request PDF

https://www.researchgate.net/publication/352267491_Ultradilute_Measurements_of_Self-Association_for_the_Identification_of_Antibodies_with_Favorable_High-Concentration_Solution_Properties

Enhanced rational protein engineering to reduce viscosity in high-concentration IgG1 antibody solutions - PubMed

https://pubmed.ncbi.nlm.nih.gov/40776898/

Enhanced rational protein engineering to reduce viscosity in high-concentration IgG1 antibody solutions - PubMed

https://pubmed.ncbi.nlm.nih.gov/40776898/

Enhanced rational protein engineering to reduce viscosity in high-concentration IgG1 antibody solutions - PubMed

https://pubmed.ncbi.nlm.nih.gov/40776898/

Prediction of Antibody Viscosity from Dilute Solution Measurements

https://www.mdpi.com/2073-4468/12/4/78

Heavy chain-only antibodies with a stabilized human VH in transgenic chickens for therapeutic antibo

https://www.omniab.com/wp-content/uploads/2025/01/Heavy-chain-only-antibodies-with-a-stabilized-human-VH-in-transgenic-chickens-for-therapeutic-antibody-discovery-1.pdf

Prediction of Antibody Viscosity from Dilute Solution Measurements

https://www.mdpi.com/2073-4468/12/4/78

Intrinsic physicochemical profile of marketed antibody-based ... - PNAS

https://www.pnas.org/doi/10.1073/pnas.2020577118
WO2018035470A1 - Assay for determining potential to self-association of a protein using concentration-dependent self-interaction nanoparticle spectroscopy - Google Patents

https://patents.google.com/patent/WO2018035470A1/en
WO2018035470A1 - Assay for determining potential to self-association of a protein using concentration-dependent self-interaction nanoparticle spectroscopy - Google Patents

https://patents.google.com/patent/WO2018035470A1/en

Intrinsic physicochemical profile of marketed antibody-based ... - PNAS

https://www.pnas.org/doi/10.1073/pnas.2020577118

Enhanced rational protein engineering to reduce viscosity in high-concentration IgG1 antibody solutions - PubMed

https://pubmed.ncbi.nlm.nih.gov/40776898/

Enhanced rational protein engineering to reduce viscosity in high-concentration IgG1 antibody solutions - PubMed

https://pubmed.ncbi.nlm.nih.gov/40776898/

Enhanced rational protein engineering to reduce viscosity in high-concentration IgG1 antibody solutions - PubMed

https://pubmed.ncbi.nlm.nih.gov/40776898/

Enhanced rational protein engineering to reduce viscosity in high-concentration IgG1 antibody solutions - PubMed

https://pubmed.ncbi.nlm.nih.gov/40776898/

Enhanced rational protein engineering to reduce viscosity in high-concentration IgG1 antibody solutions - PubMed

https://pubmed.ncbi.nlm.nih.gov/40776898/

Rapid and accurate in silico solubility screening of a monoclonal antibody library | Scientific Reports

https://www.nature.com/articles/s41598-017-07800-w?error=cookies_not_supported&code=a35da5f5-77ff-433f-a703-3929b1c47d37

Rapid and accurate in silico solubility screening of a monoclonal antibody library | Scientific Reports

https://www.nature.com/articles/s41598-017-07800-w?error=cookies_not_supported&code=a35da5f5-77ff-433f-a703-3929b1c47d37

Ultradilute Measurements of Self-Association for the Identification of Antibodies with Favorable High-Concentration Solution Properties | Request PDF

https://www.researchgate.net/publication/352267491_Ultradilute_Measurements_of_Self-Association_for_the_Identification_of_Antibodies_with_Favorable_High-Concentration_Solution_Properties

Viscosity Prediction of High-Concentration Antibody Solutions with ...

https://pubs.acs.org/doi/10.1021/acs.jcim.3c00947

Viscosity Prediction of High-Concentration Antibody Solutions with ...

https://pubs.acs.org/doi/10.1021/acs.jcim.3c00947

Viscosity Prediction of High-Concentration Antibody Solutions with ...

https://pubs.acs.org/doi/10.1021/acs.jcim.3c00947

Low-data interpretable deep learning prediction of antibody viscosity using a biophysically meaningful representation | Scientific Reports

https://www.nature.com/articles/s41598-023-28841-4?error=cookies_not_supported&code=b455399a-abfa-42a2-8fa7-b88a6af1730a

Low-data interpretable deep learning prediction of antibody viscosity using a biophysically meaningful representation | Scientific Reports

https://www.nature.com/articles/s41598-023-28841-4?error=cookies_not_supported&code=b455399a-abfa-42a2-8fa7-b88a6af1730a

Low-data interpretable deep learning prediction of antibody viscosity using a biophysically meaningful representation | Scientific Reports

https://www.nature.com/articles/s41598-023-28841-4?error=cookies_not_supported&code=b455399a-abfa-42a2-8fa7-b88a6af1730a

Low-data interpretable deep learning prediction of antibody viscosity using a biophysically meaningful representation | Scientific Reports

https://www.nature.com/articles/s41598-023-28841-4?error=cookies_not_supported&code=b455399a-abfa-42a2-8fa7-b88a6af1730a

Low-data interpretable deep learning prediction of antibody viscosity using a biophysically meaningful representation | Scientific Reports

https://www.nature.com/articles/s41598-023-28841-4?error=cookies_not_supported&code=b455399a-abfa-42a2-8fa7-b88a6af1730a

Machine learning prediction of antibody aggregation and viscosity for high concentration formulation development of protein therapeutics - PubMed

https://pubmed.ncbi.nlm.nih.gov/35075980/

Machine learning prediction of antibody aggregation and viscosity for high concentration formulation development of protein therapeutics - PubMed

https://pubmed.ncbi.nlm.nih.gov/35075980/

Machine learning prediction of antibody aggregation and viscosity for high concentration formulation development of protein therapeutics - PubMed

https://pubmed.ncbi.nlm.nih.gov/35075980/
Leveraging Multi-Modal Feature Learning for Predictions of Antibody ...

https://www.biorxiv.org/content/10.1101/2024.08.30.610550v1.full-text

ProtT5 and random forests-based viscosity prediction method for ...

https://www.sciencedirect.com/science/article/pii/S0928098724000162

ProtT5 and random forests-based viscosity prediction method for ...

https://www.sciencedirect.com/science/article/pii/S0928098724000162
Reduction of monoclonal antibody viscosity using interpretable ...

https://www.tandfonline.com/doi/full/10.1080/19420862.2024.2303781

Machine learning prediction of antibody aggregation and viscosity for high concentration formulation development of protein therapeutics - PubMed

https://pubmed.ncbi.nlm.nih.gov/35075980/

Rapid and accurate in silico solubility screening of a monoclonal antibody library | Scientific Reports

https://www.nature.com/articles/s41598-017-07800-w?error=cookies_not_supported&code=a35da5f5-77ff-433f-a703-3929b1c47d37

Rapid and accurate in silico solubility screening of a monoclonal antibody library | Scientific Reports

https://www.nature.com/articles/s41598-017-07800-w?error=cookies_not_supported&code=a35da5f5-77ff-433f-a703-3929b1c47d37

Rapid and accurate in silico solubility screening of a monoclonal antibody library | Scientific Reports

https://www.nature.com/articles/s41598-017-07800-w?error=cookies_not_supported&code=a35da5f5-77ff-433f-a703-3929b1c47d37

A systematic review of commercial high concentration antibody drug products approved in the US: formulation composition, dosage form design and primary packaging considerations - PMC

https://pmc.ncbi.nlm.nih.gov/articles/PMC10228404/

A systematic review of commercial high concentration antibody drug products approved in the US: formulation composition, dosage form design and primary packaging considerations - PMC

https://pmc.ncbi.nlm.nih.gov/articles/PMC10228404/

A systematic review of commercial high concentration antibody drug products approved in the US: formulation composition, dosage form design and primary packaging considerations - PMC

https://pmc.ncbi.nlm.nih.gov/articles/PMC10228404/

A systematic review of commercial high concentration antibody drug products approved in the US: formulation composition, dosage form design and primary packaging considerations - PMC

https://pmc.ncbi.nlm.nih.gov/articles/PMC10228404/

A systematic review of commercial high concentration antibody drug products approved in the US: formulation composition, dosage form design and primary packaging considerations - PMC

https://pmc.ncbi.nlm.nih.gov/articles/PMC10228404/

A systematic review of commercial high concentration antibody drug products approved in the US: formulation composition, dosage form design and primary packaging considerations - PMC

https://pmc.ncbi.nlm.nih.gov/articles/PMC10228404/

A systematic review of commercial high concentration antibody drug products approved in the US: formulation composition, dosage form design and primary packaging considerations - PMC

https://pmc.ncbi.nlm.nih.gov/articles/PMC10228404/

A systematic review of commercial high concentration antibody drug products approved in the US: formulation composition, dosage form design and primary packaging considerations - PMC

https://pmc.ncbi.nlm.nih.gov/articles/PMC10228404/

A systematic review of commercial high concentration antibody drug products approved in the US: formulation composition, dosage form design and primary packaging considerations - PMC

https://pmc.ncbi.nlm.nih.gov/articles/PMC10228404/

A systematic review of commercial high concentration antibody drug products approved in the US: formulation composition, dosage form design and primary packaging considerations - PMC

https://pmc.ncbi.nlm.nih.gov/articles/PMC10228404/

A systematic review of commercial high concentration antibody drug products approved in the US: formulation composition, dosage form design and primary packaging considerations - PMC

https://pmc.ncbi.nlm.nih.gov/articles/PMC10228404/

A systematic review of commercial high concentration antibody drug products approved in the US: formulation composition, dosage form design and primary packaging considerations - PMC

https://pmc.ncbi.nlm.nih.gov/articles/PMC10228404/

A systematic review of commercial high concentration antibody drug products approved in the US: formulation composition, dosage form design and primary packaging considerations - PMC

https://pmc.ncbi.nlm.nih.gov/articles/PMC10228404/

A systematic review of commercial high concentration antibody drug products approved in the US: formulation composition, dosage form design and primary packaging considerations - PMC

https://pmc.ncbi.nlm.nih.gov/articles/PMC10228404/

A systematic review of commercial high concentration antibody drug products approved in the US: formulation composition, dosage form design and primary packaging considerations - PMC

https://pmc.ncbi.nlm.nih.gov/articles/PMC10228404/
WO2018035470A1 - Assay for determining potential to self-association of a protein using concentration-dependent self-interaction nanoparticle spectroscopy - Google Patents

https://patents.google.com/patent/WO2018035470A1/en

A systematic review of commercial high concentration antibody drug products approved in the US: formulation composition, dosage form design and primary packaging considerations - PMC

https://pmc.ncbi.nlm.nih.gov/articles/PMC10228404/

A systematic review of commercial high concentration antibody drug products approved in the US: formulation composition, dosage form design and primary packaging considerations - PMC

https://pmc.ncbi.nlm.nih.gov/articles/PMC10228404/

Manufacturing challenges with high concentration biologics | Cytiva

https://www.cytivalifesciences.com/en/us/solutions/bioprocessing/knowledge-center/hcd-challenges

Manufacturing challenges with high concentration biologics | Cytiva

https://www.cytivalifesciences.com/en/us/solutions/bioprocessing/knowledge-center/hcd-challenges

Manufacturing challenges with high concentration biologics | Cytiva

https://www.cytivalifesciences.com/en/us/solutions/bioprocessing/knowledge-center/hcd-challenges

Manufacturing challenges with high concentration biologics | Cytiva

https://www.cytivalifesciences.com/en/us/solutions/bioprocessing/knowledge-center/hcd-challenges

Manufacturing challenges with high concentration biologics | Cytiva

https://www.cytivalifesciences.com/en/us/solutions/bioprocessing/knowledge-center/hcd-challenges

Manufacturing challenges with high concentration biologics | Cytiva

https://www.cytivalifesciences.com/en/us/solutions/bioprocessing/knowledge-center/hcd-challenges

Manufacturing challenges with high concentration biologics | Cytiva

https://www.cytivalifesciences.com/en/us/solutions/bioprocessing/knowledge-center/hcd-challenges

Manufacturing challenges with high concentration biologics | Cytiva

https://www.cytivalifesciences.com/en/us/solutions/bioprocessing/knowledge-center/hcd-challenges

Manufacturing challenges with high concentration biologics | Cytiva

https://www.cytivalifesciences.com/en/us/solutions/bioprocessing/knowledge-center/hcd-challenges

Manufacturing challenges with high concentration biologics | Cytiva

https://www.cytivalifesciences.com/en/us/solutions/bioprocessing/knowledge-center/hcd-challenges

Manufacturing challenges with high concentration biologics | Cytiva

https://www.cytivalifesciences.com/en/us/solutions/bioprocessing/knowledge-center/hcd-challenges

Manufacturing challenges with high concentration biologics | Cytiva

https://www.cytivalifesciences.com/en/us/solutions/bioprocessing/knowledge-center/hcd-challenges

Manufacturing challenges with high concentration biologics | Cytiva

https://www.cytivalifesciences.com/en/us/solutions/bioprocessing/knowledge-center/hcd-challenges

A systematic review of commercial high concentration antibody drug products approved in the US: formulation composition, dosage form design and primary packaging considerations - PMC

https://pmc.ncbi.nlm.nih.gov/articles/PMC10228404/

A systematic review of commercial high concentration antibody drug products approved in the US: formulation composition, dosage form design and primary packaging considerations - PMC

https://pmc.ncbi.nlm.nih.gov/articles/PMC10228404/

Mechanistic and predictive formulation development for viscosity ...

https://pmc.ncbi.nlm.nih.gov/articles/PMC12439579/

A systematic review of commercial high concentration antibody drug products approved in the US: formulation composition, dosage form design and primary packaging considerations - PMC

https://pmc.ncbi.nlm.nih.gov/articles/PMC10228404/

High throughput cross-interaction measures for human IgG1 ...

https://pmc.ncbi.nlm.nih.gov/articles/PMC4622737/

Rapid and accurate in silico solubility screening of a monoclonal antibody library | Scientific Reports

https://www.nature.com/articles/s41598-017-07800-w?error=cookies_not_supported&code=a35da5f5-77ff-433f-a703-3929b1c47d37

Self Interaction assays. The AC-SINS assay correlated significantly...

https://www.researchgate.net/figure/Self-Interaction-assays-The-AC-SINS-assay-correlated-significantly-with-CSI-BLI-response_fig1_277777725

Heavy chain-only antibodies with a stabilized human VH in transgenic chickens for therapeutic antibo

https://www.omniab.com/wp-content/uploads/2025/01/Heavy-chain-only-antibodies-with-a-stabilized-human-VH-in-transgenic-chickens-for-therapeutic-antibody-discovery-1.pdf

Machine learning prediction of antibody aggregation and viscosity for high concentration formulation development of protein therapeutics - PubMed

https://pubmed.ncbi.nlm.nih.gov/35075980/